\section{Product perspective}
The services are going to be used by Third party users interested in having access to statistical, individual data or in organizing competitions. In order to provide those services there will be a web page and, mobile application and a server application.  \\
Individual users interact with the system via mobile application while third-party users interact via web page. \\
All services provided by Data4Help are based on the collection of data associated to an individual user: since his registration, data coming from user’s device are stored in a database on the server and continuously updated. \\
API documentation for Third-parties use will be available on the web page. \\
Every entity, registered as Third party, can request statistics and, if Data4Help properly anonymize them, those data are available on the web page.   
Third-party users need previous User’ authorization to access individual data. A User gives or denies authorization through Data4Help mobile application. Individual data are accessible on the web page to authorized Third-party users.  \\
Real time individual data are accessible on the web page to subscribed third-party users: the authorization to the subscription must be previously given by the User on the mobile application.  \\
Track4Run competitions’ organizers need to register as Third-party on the web page. \\
The web page provides organizers with all the features they need to arrange, delete, schedule and follow the run in real time from a map. 
\includepdf[pages=-]{sections/overview/UML_RASD.pdf}
%-------------------------------------------------------------------------------------------------------------
\section{Product functions}
\subsection{Data management}
By mobile application, individual users upon registration will provide their personal data (name, age, gender, address, ID, weight, height) into the system. Data4Help will grant access to their personal and collected data to third-party users if requested anonymously. If not, it will warn the individual user that a third-party user is questing access and ask for their approval; furthermore, the system will keep a record of third-party users subscribed to their data and it will allow the individual user to cancel the approval.
\subsection{Data access}
The Data4Help system upon registration of third-party users will require their ID and then will allow them to request saved data on individual users. Data4Help will allow them to request access to data on a specific individual user after providing a the ID of the individual user; it will notify them after the request is sent and when the user decides upon its approval. \\
Third-party users will be able also to request anonymized data by specifying a group of people according to relevant filters as age, weight, geographical area, etc. 
Data4Help will analyse the request and decide whether it is possible to properly anonymize the requested data and will notify the third-party upon its decision. If possible, will grant access to the requested data. Third-party users will be able to interact with the system by web interface or using the system APIs.
\subsection{Data subscription}
The Data4Help system will allow registered third-party users to subscribe to data on users. This service will provide data as soon as it is produced to the third-party without the need for constants approvals.
In case a third-party subscribe to have live data for a specific user the system will act as with a stored data. If the user cancels a previously granted access Data4Help will warn the third-party users that had access to those data. Third-party users will be able to request data by web interface or using the released Data4Help APIs.
\\
\subsection{Emergency call}
The Data4Help system will be able to forward a request for help to the Emergency Service within 5 seconds from the moment AutomatedSOS detects a dangerous drop in health parameters values. The Emergency Service takes charge of the request and sends an ambulance. 
\\
\subsection{Running competition organization}
The Track4Run system will allow users to organize running competition. It will require to a RO: 
\begin{itemize}
\item time of the competition
\item path of the competition
\item restrictions on participants 
\item an optional message for who wants to enroll 
\end{itemize}
The system provides the RO with data on the enrolled users.
\subsection{Running competition monitoring}
The Track4Run system will allow users to monitor a running competition; it is mandatory for a RS to subscribe to the run as a spectator. The system will provide the RS with names and real-time position of all the participants.
\subsection{Running competition enrolling}
The Track4Run system will allow users to enrol to an already existing competition. Track4Run will provide participants them with the final ranking and their personal data throughout the run: rate per kilometer, instantaneous speed, the missing and ran distance, the position.

%-------------------------------------------------------------------------------------------------------------
\section{User charateristics}
\begin{itemize}
\item	\textit{Visitor}: a person, third-party, that has not registered in yet and has access only to registration;
\item	\textit{User}:  a person, third-party or user, that has registered;
\item	\textit{Individual user}: every registered person from whom the system collects location and health data. The user has access to all data collected since his registration and to the AutomatedSOS and Track4Run services;
\item	\textit{Run participant}: an individual who wants to enrol and take part in a run;
\item	\textit{Run spectator}: an individual user who wants to spectate a run, namely to see all information on the run;
\item \textit{Third-party user}: every entity registered with the purpose to request data from Data4Help for external use or to organize a running competition exploiting Track4Run services;
\item	\textit{Run organizer}: third-party users that wants to organize a run and receive all information on it and the participants
\item	\textit{Emergency service}: external participant that receives and take charge of the request for an ambulance.
\end{itemize}
%-------------------------------------------------------------------------------------------------------------

\section{Assumptions}
\begin{itemize}
\item	[\textbf{D1}] The age, height, weight, address, gender and name provided by the user on themselves are correct
\item	[\textbf{D2}] The ID is unique
\item	[\textbf{D3}] Email address in unique
\item	[\textbf{D4}] The email is currently in use
\item	[\textbf{D5}] The kinds acd accuracy of data collected on the users’ health conditions is comprehensive and enough to determine their health status and their location whithin 10 meters
\item	[\textbf{D6}] The user has wearable device connected to a smartphone
\item	[\textbf{D7}] The device can successfully contact the Emergency Service 24/7
\item	[\textbf{D8}] Emergency Service takes charge of every request sending an ambulance
\item	[\textbf{D9}] The ROs organize only authorized running competitions respecting laws and common sense
\end{itemize}

%-------------------------------------------------------------------------------------------------------------
\section{Constraints}
\begin{itemize}
\item IUs' personal data and GPS signal must be aquired with their consent
\item IUs' personal data must be stored and transmitted safely. 
\item Data coming from user’s device are collected since his registration, stored in a server and continuously updated.
\end{itemize}
%-------------------------------------------------------------------------------------------------------------
\section{Dependencies}
Due the nature of the project the system can be diveded in three fairly indipendent systems: Data4Help, AutomatedSos and Track4Run. ASOS and T4R are compleatly dependent on D4H, regarding the collection of data and the registration and login processes. Specifically ASOS requires the D4H mobile system to collect data to use while
T4R requires a completed D4H.
