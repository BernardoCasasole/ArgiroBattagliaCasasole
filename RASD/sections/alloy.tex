\section{Data4Help}
The first alloy model focus on the lifecycle of data requests performed by TPUs on IUs, on the access from TPUs to approved data and their being blocked, namely:
\begin{itemize}
\item a request must be created waiting for approval and must then go in a deny or approve state, after which it never changes state, unless it is a subscription request whose approval can be revocked, causing the request to go from an approval state ti a deny state;
\item a TPU has access only to data for which exists either a stored data request approved after the data was collected or a subscription request approved at the time the data was produced
\item a TPU cannot make a request for data of an individual user if it is blocked
\end{itemize}
To do so we considered a time frame for which the requests have the set of state they were in every instant. 
The IUs have instead the set of blocked TPUs; this is considered already filled with the TPUs that the IU blocked and never change in the time frame.
The TPUs have a set of data it has access to, after the time frame.
\subsection{Alloy model}
\input{sections/alloy/alloyD4H}
\subsection{Generated world}
\includegraphics[width = 0.8\textheight, angle=90]{sections/alloy/GenWorldD4H.png}
\subsection{Alloy Result}
\includegraphics[height = 0.8\textheight]{sections/alloy/resultD4H.png}
\clearpage
\section{AutomatedSOS}
The second alloy model focus on the lifecycle of an emergency call:
\begin{itemize}
\item a call must end after it has contacted the ES or after it was blocked by the IU
\item for every time interval in which the health was critical exactly one call must have been made
\end{itemize}
For simplicity sake, this model considers that an abulance is called every time the status drops ti critical.
\subsection{Alloy model}
\begin{lstlisting}
open util/time



/* Signatures
*******************************************************************/
abstract sig User{
	id: one Int,
	locationSet: set Location -> Time,
	dataSet: set Data -> Time,
	referenceData: set Data /*Set of data for critical health condition*/
}{
	#(referenceData) > 0
	all t: Time | one data: Data | dataSet.t = data
}
sig Data{}
sig Location{}

sig EmergencyCall{
	user: one User,
	location: one Location,
	callTime: one Time,
	stateSet: set CallState one -> Time
}{
	location = user.locationSet.callTime
	all t: Time | one cs: CallState | stateSet.t = cs
}
enum CallState{N, C, B, D}
//NOT_STARTED, CALLING, BLOCKED, DONE

/* Facts
*******************************************************************/
fact idUnique{
	no disjoint u1, u2: User | u1.id = u2.id
}

fact callStates{
	/*A state is defined for every time istant after callTime*/
	all ec: EmergencyCall | all t: Time | one state: CallState |
		 state in ec.stateSet.t
	/*Created in "Calling"*/
	all ec: EmergencyCall |
		(ec.stateSet.(ec.callTime) = C) && (no t: Time | (lt[t, ec.callTime]) && (ec.stateSet.t = C))
	/*When the call is "not started", it has never changed state*/
	all ec: EmergencyCall | all t1, t2: Time | 
		(gte[t2, t1] && ec.stateSet.t2 = N) => (ec.stateSet.t1 = N)
	/*Once the call is "Blocked" it can't change state again*/
	 all ec: EmergencyCall | all t1, t2: Time | 
		(gte[t2, t1] && ec.stateSet.t1 = B) => (ec.stateSet.t2 = B)
	/*Once the call is "Done" it can't change state again*/
	 all ec: EmergencyCall | all t1, t2: Time | 
		(gte[t2, t1] && ec.stateSet.t1 = D) => (ec.stateSet.t2 = D)
	
}/*Allowed state sequences: {N -> C -> B}{N -> C -> D}*/

/*For every instant in wich the user health went critical a call has been made when health dropped*/
fact whenCriticalThereIsACall{
	all us: User | all t: Time | 
	(us.dataSet.t in us.referenceData) => (
	one ec: EmergencyCall | all t1: Time |
		(ec.user = us) &&
		(lte[ec.callTime, t]) &&
		((gte[t1, ec.callTime] && lte[t1, t]) => (us.dataSet.t1 in us.referenceData))
	)
}

/*No calls are made when health is not critical*/
fact {
	all ec: EmergencyCall  | one us: User | 
	(ec.user = us) &&
	(us.dataSet.(ec.callTime) in us.referenceData)
}

/* Predicates
*******************************************************************/
pred callAmbulance[us: User, t: Time]{
	//pre-conditions
	//post-conditions
	one ec: EmergencyCall |
		(ec.user = us) &&
		(ec.callTime = t.next) &&
		(ec.stateSet.(t.next) = C)
}

pred completeCall[ec: EmergencyCall, t: Time]{
	//pre-conditions
	ec.stateSet.t = C
	//post-conditions
	ec.stateSet.(t.next) = D
}

pred blockCall[ec: EmergencyCall, t: Time]{
	//pre-conditions
	ec.stateSet.t = C
	//post-conditions
	ec.stateSet.(t.next) = B
}

pred show {
	 #(User) = 2
	#(EmergencyCall) = 2
	(all us: User | some t: Time | callAmbulance[us, t])
	(some ec: EmergencyCall | some t: Time | completeCall[ec, t])
	(some ec: EmergencyCall | some t: Time | blockCall[ec, t])
}

/* Run
*******************************************************************/
run callAmbulance for 5
run completeCall for 5
run blockCall for 5

run show for 5


\end{lstlisting}
\subsection{Generated world}
\includegraphics[width = 0.8\textheight, angle=90]{sections/alloy/GenWorldASOS.png}
\subsection{Alloy Result}
\includegraphics[height = 0.8\textheight]{sections/alloy/resultASOS.png}
\clearpage
\section{Track4Run}
The third alloy model focus on the lifecycle of a run and the restriction on participants:
\begin{itemize}
\item every organized run must either be canceled before they start or successfuly end;
\item IUs can enroll to participate or spectate only during specific state of the run, respectively when the run has been organized but not started and when the run has started
\end{itemize}
For simplicity sake, this model considers that an abulance is called every time the status drops ti critical.
\subsubsection{Alloy model}
\input{sections/alloy/alloyT4R}
\subsection{Generated world}
\includegraphics[width = 0.8\textheight, angle=90]{sections/alloy/GenWorldT4R.png}
\subsection{Alloy Result}
\includegraphics[height = 0.8\textheight]{sections/alloy/resultT4R.png}