\section{Data4Help}
\subsection{Alloy model}
The requests contain the set of state they were in the time frame considered.
The blocked TPUs are fixed and never change in the time frame.
The data accessible from the TPUs is the data is has access to after the time frame.
\input{sections/alloy/alloyD4H}
\subsection{Generated world}
\includegraphics[width = \textheight, angle=90]{sections/alloy/GenWorldD4H.png}
\subsection{Alloy Result}
\includegraphics[height = \textheight]{sections/alloy/resultD4H.png}

\section{AutomatedSOS}
\subsection{Alloy model}
For simplicity sake, this model considers that an abulance is called every time the status drops ti critical, not that it waits to be reactivated 
every time it makes a call
\begin{lstlisting}
open util/time



/* Signatures
*******************************************************************/
abstract sig User{
	id: one Int,
	locationSet: set Location -> Time,
	dataSet: set Data -> Time,
	referenceData: set Data /*Set of data for critical health condition*/
}{
	#(referenceData) > 0
	all t: Time | one data: Data | dataSet.t = data
}
sig Data{}
sig Location{}

sig EmergencyCall{
	user: one User,
	location: one Location,
	callTime: one Time,
	stateSet: set CallState one -> Time
}{
	location = user.locationSet.callTime
	all t: Time | one cs: CallState | stateSet.t = cs
}
enum CallState{N, C, B, D}
//NOT_STARTED, CALLING, BLOCKED, DONE

/* Facts
*******************************************************************/
fact idUnique{
	no disjoint u1, u2: User | u1.id = u2.id
}

fact callStates{
	/*A state is defined for every time istant after callTime*/
	all ec: EmergencyCall | all t: Time | one state: CallState |
		 state in ec.stateSet.t
	/*Created in "Calling"*/
	all ec: EmergencyCall |
		(ec.stateSet.(ec.callTime) = C) && (no t: Time | (lt[t, ec.callTime]) && (ec.stateSet.t = C))
	/*When the call is "not started", it has never changed state*/
	all ec: EmergencyCall | all t1, t2: Time | 
		(gte[t2, t1] && ec.stateSet.t2 = N) => (ec.stateSet.t1 = N)
	/*Once the call is "Blocked" it can't change state again*/
	 all ec: EmergencyCall | all t1, t2: Time | 
		(gte[t2, t1] && ec.stateSet.t1 = B) => (ec.stateSet.t2 = B)
	/*Once the call is "Done" it can't change state again*/
	 all ec: EmergencyCall | all t1, t2: Time | 
		(gte[t2, t1] && ec.stateSet.t1 = D) => (ec.stateSet.t2 = D)
	
}/*Allowed state sequences: {N -> C -> B}{N -> C -> D}*/

/*For every instant in wich the user health went critical a call has been made when health dropped*/
fact whenCriticalThereIsACall{
	all us: User | all t: Time | 
	(us.dataSet.t in us.referenceData) => (
	one ec: EmergencyCall | all t1: Time |
		(ec.user = us) &&
		(lte[ec.callTime, t]) &&
		((gte[t1, ec.callTime] && lte[t1, t]) => (us.dataSet.t1 in us.referenceData))
	)
}

/*No calls are made when health is not critical*/
fact {
	all ec: EmergencyCall  | one us: User | 
	(ec.user = us) &&
	(us.dataSet.(ec.callTime) in us.referenceData)
}

/* Predicates
*******************************************************************/
pred callAmbulance[us: User, t: Time]{
	//pre-conditions
	//post-conditions
	one ec: EmergencyCall |
		(ec.user = us) &&
		(ec.callTime = t.next) &&
		(ec.stateSet.(t.next) = C)
}

pred completeCall[ec: EmergencyCall, t: Time]{
	//pre-conditions
	ec.stateSet.t = C
	//post-conditions
	ec.stateSet.(t.next) = D
}

pred blockCall[ec: EmergencyCall, t: Time]{
	//pre-conditions
	ec.stateSet.t = C
	//post-conditions
	ec.stateSet.(t.next) = B
}

pred show {
	 #(User) = 2
	#(EmergencyCall) = 2
	(all us: User | some t: Time | callAmbulance[us, t])
	(some ec: EmergencyCall | some t: Time | completeCall[ec, t])
	(some ec: EmergencyCall | some t: Time | blockCall[ec, t])
}

/* Run
*******************************************************************/
run callAmbulance for 5
run completeCall for 5
run blockCall for 5

run show for 5


\end{lstlisting}
\subsection{Generated world}
\includegraphics[width = \textheight, angle=90]{sections/alloy/GenWorldASOS.png}
\subsection{Alloy Result}
\includegraphics[height = \textheight]{sections/alloy/resultASOS.png}

\section{Track4Run}
\subsubsection{Alloy model}
\input{sections/alloy/alloyT4R}
\subsection{Generated world}
\includegraphics[width = \textheight, angle=90]{sections/alloy/GenWorldT4R.png}
\subsection{Alloy Result}
\includegraphics[height = \textheight]{sections/alloy/resultT4R.png}