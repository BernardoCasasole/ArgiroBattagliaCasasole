\section{Purpose}
\subsection{Goals}
\begin{itemize}
\item
[\textbf{G1}] Allow users to properly use the services they wish to employ
\item
[\textbf{G2}] Allow visitor to register as individual or third-party user
\item
[\textbf{G3}] Allow individual users to monitor their location and health parameters
\item
[\textbf{G4}] Allow third-party users to request the data on specific users
\item
[\textbf{G5}] Allow individual users to approve or deny the specific request for their data
\item
[\textbf{G6}] Allow third-party users to request data on anonymized groups of individual users
\item
[\textbf{G7}] Call an ambulance if the system detects a critical health condition
\item
[\textbf{G8}] Allow third-party users to organize running competitions
\item
[\textbf{G9}] Allow Individual users to enrol in existing running competitions as participants 
\item
[\textbf{G10}] Allow Individual users to subscribe in existing running competition as spectators to monitor underway competitions
\end{itemize}


%-------------------------------------------------------------------------------------------------------------
\section{Scope}
The service Data4Help is designed to monitor the location and health status of individual registered to it using wearable devices. Upon registration individual users agree to the acquisition and usage of data by TrackMe and will be able to do see their own data.
The collected data is available to be requested by registered third-party users. The first possibility is requesting data on a specific individual user, for which it is required an identification code of the individual user(Social security number, fiscal code, etc.); the second possibility is accessing anonymized data on groups of individual users, so the system will have to allow third-party users to filter the users by age, geographic area, weight, etc. while keeping the data properly anonymized.
For both possibilities third-party users can simply acquire the data stored or subscribe to receive the data as soon as it is produced.
\\
\\
AutomatedSOS is a service designed to provide emergency health support for elderly people, to be integrated on top of Data4Help. Those individual users’ parameters about blood pressure, heartbeat and blood oxygenation are constantly checked: whenever they represent a critical health condition, AutomatedSOS, within 5 seconds from the drop of those values, contacts an emergency service asking to send an ambulance at the user location.
\\
\\
The main purpose of the Track4Run is to support running competition organizer and jogging lovers in convening great sport events with little effort. Namely Track4Run is a service-to-be focused on organizing running competition and on monitoring competitors. A user of Track4Run will also have to be a user of Data4Help. 
\\ 
\\
Third-party users will be able to: 
\begin{itemize}
\item 
organize a running competition specifying time, path of the competition, restrictions on participants and message for who wants to enroll;
\item 
see the participants; 
\item 
cancel a run they have organized. 
\end{itemize}

Individual users will be able to: 
\begin{itemize}
\item 
discover new nearby runs upon their creation by an app notification; 
\item 
be notified if a run they are participants in is canceled; 
\item 
search for organized runs and register to participate in a future run or spectate an ongoing run.
\end{itemize}
During the run: 
\begin{itemize}
\item 
organizer, participants and spectators will be able to see the name and positions of the participants; 
\item 
the system must be able to distinguish between actual participants and individual users who registered to the run but are not competing, simply checking if their position is on the run path. 
\end{itemize}

%-------------------------------------------------------------------------------------------------------------
\section{Definitions}
\begin{itemize}
\item \textit{Visitor}:someone that has not registered in yet;
\item \textit{User}: a person, third-party or user, that has registered;
\item \textit{Individual User}: every registered person from whom the system collects data; 
\item \textit{Run Participant}: an individual who wants to take part in a run;
\item \textit{Run Spectator}: an individual user who wants to spectate a run;
\item \textit{Third-Party User}:every entity registered with the purpose to request data for external use;
\item \textit{Run Organizer}: third-party users that wants to organize a run;
\item \textit{Emergency Service}: external participant that receives and take charge of the request for an ambulance;
\item \textit{Identification code}: a unique code for individual users or third-party users which identify them; for individual users it can be the Social security number, fiscal code, etc.; for third-party users it can be also the VAT number.
\item \textit{Stored Data}: the data on a IU collected so far.
\item \textit{Data Request}: a request for data made from a TPU.
\item \textit{Stored Data Request}: a data request for stored data.
\item \textit{Subscription Request}: a request for subscribing to newly generated data.
\end{itemize}
\section{Acronyms}
\begin{itemize}
\item RASD: Requirement Analysis and Specification Document. 
\item API: Application Programming Interface
\item TPU: Third-party User
\item RO: Run Organizer 
\item	RS: Run Spectator 
\item	RP: Run Participant 
\item	ES: Emergency Service 
\item	ID: Identification code
\item	D4H: Data4Help
\item	ASOS: AutomatedSOS
\item	T4R: Track4Run
\end{itemize}

\section{Abbreviations}
\begin{itemize}
\item Gn: n-goal
\item Dn: n-Domain assumption
\item Rn: n-Requirement
\end{itemize}

%-------------------------------------------------------------------------------------------------------------
\section{Revision history}
%-------------------------------------------------------------------------------------------------------------
\section{Reference Documents }
%-------------------------------------------------------------------------------------------------------------
\section{Document Structure}

