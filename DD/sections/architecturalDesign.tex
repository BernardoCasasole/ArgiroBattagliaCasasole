\section{Overview}
The architecture style used is a client/server structure with multiple tiers while the backbone will handle the dispatch of live data through the system. The presentation layer will be hosted on both clients (IUs and TPUs clients) while the application server will host the logic layer and the database server the data layer. The IU client is going to be a thick client, hosting a branch of the application logic to handle better and faster the system functionalities.
%high level---------------------------------------------------------------------------------------------------
\subsection{High level components and basic interactions}
\begin{figure}[H]
\caption{High level components' elements}
\includegraphics[width = \textwidth]{sections/architecturalDesign/overview_1.jpg}
\end{figure}

The overall structure, at high level, is made of three main components and their interaction. 
The red component refers to the tools the individual user needs to interface with Data4Help System, it communicates with the Application Server that is part of the blue component charged with the Server Architecture. This is composed by a Database Server that includes the DBMS, a Mail Server which means to exchange SMTP messages with other Mail Servers (external to the system), an Application Server communicating with any other element in the Server Architecture, a Web Server and a Proxy (meant to dispatch requests to Application and Web Servers). 
The proxy links the Server Architecture with the green component charged with the interaction with the third party user that takes place through Data4Help Web Page. 

\subsection{Interaction between Server Architecture and Individual User}
\begin{figure}[H]
\caption{Mobile App connections}
\includegraphics[width = \textwidth]{sections/architecturalDesign/overview_individual_user.jpg}
\end{figure}

The Mobile App receives data from the smartwatch, exchanges informations with the Individual User and communicates with the Application Server: at different levels are specified protocols that are supposed to be used. 
 

\subsection{Interaction between Server Architecture and Third-Party User}
\begin{figure}[H]
\caption{Connection between Web page and Servers}
\includegraphics[width = \textwidth ,  height=8cm]{sections/architecturalDesign/overview_web.jpg}
\end{figure}

The browser hosting the Web Page needs to communicate both with the Web Server and the Application Server. 
The Web Server can easily handle and exchange HTML, CSS and JavaScript files with the client; the Application Server manages methods like GET, POST receiving a REST API call and forwarding data in JSON format. Data to forward are provided by the Database Server which includes the DBMS: a request in XML is sent by the Application Server, the DBMS processes the request and extracts data from the database that are sent back to the Application Server in a XML file. To establish a communication channel between the Application Server and the Web Server is not a necessity, however it provides an alternative to REST API: developers are up to decide to implement them both or to keep the REST API alone. 

%-------------------------------------------------------------------------------------------------------------
%-------------------------------------------------------------------------------------------------------------
%-------------------------------------------------------------------------------------------------------------
%-------------------------------------------------------------------------------------------------------------
%-------------------------------------------------------------------------------------------------------------
\section{Component view}
The system is divided in four subsystem:
\begin{itemize}
\item \textbf{\href{subsect:backboneComponentView}{Backbone}}
\item \textbf{\href{subsect:D4HComponentView}{Data4Help}}
\item \textbf{\href{subsect:ASOSComponentView}{AutomatedSOS}}
\item \textbf{\href{subsect:T4RComponentView}{Track4Run}}
\end{itemize}
The Backbone is the core of the system: all other subsystems interact with it and don't interact with each other. The backbone provides interfaces for authentication and to receive live data published form the Backbone with a event-based paradigm.\\
The last three are divided, on the Application server, in a router that provide an interface gathering all the subsystem functionalities, and a module, containing all other components of the subsystem, which uses the exposed method of the DBMS to be able to work indipendently. \\
On the IU and TPU clients the view component represent the presentation layer of the system, which Users can access directly.\\
The relation between the components and the model il further defined in figure \ref{fig:ModelInteractionDiagram}.
%BackBone---------------------------------------------------------------------------------------------------
\subsection{Backbone}
\label{subsect:backboneComponentView}
\begin{figure}[H]
\caption{Backbone Component View}
\includegraphics[width = \textwidth]{sections/architecturalDesign/BackboneDiagram.png}
\end{figure}
This is the backbone of the system: collects the data on the device, keep it syncronized though the system, stores it onto the database and provide the functionalities to receive live data; Furthermore provide functionality concernig authentication.
\paragraph{\textit{Data collector/dispatcher}} Allow subscribtion from other components on the IU client and publishes/dispatches the collected live data of the Individual User logged in from the device. 
\paragraph{\textit{Autenthication}} Offers services related to User authentication and the functionalities to handle their info.
\paragraph{\textit{Data Dispatcher}} Allow subscribtion from other components on the application server and publishes/dispatches the collected live data of all Users and it stores it onto the database.
%D4H--------------------------------------------------------------------------------------------------------
\subsection{Data4Help}
\label{subsect:D4HComponentView}
\begin{figure}[H]
\caption{Data4Help Component View}
\includegraphics[width = \textwidth]{sections/architecturalDesign/D4HDiagram.png}
\end{figure}
\paragraph{\textit{D4H router}} Validate the requests received from the client and dispatch them to the corresponding module or component.
\paragraph{\textit{Data Request Manager}} Provides functionality to create, approve, deny requests, block users and provide the relative data; Anonymity Evaluator is responsible to check anonymity constraints. 
%ASOS------------------------------------------------------------------------------------------------------
\subsection{AutomatedSOS}
\label{subsect:ASOSComponentView}
\begin{figure}[H]
 \caption{AutomatedSOS Component View}
\centering
\includegraphics[width = \textwidth]{sections/architecturalDesign/ASOSDiagram.png}
\end{figure}
\paragraph{\textit{ASOS router}} Validate the requests received from the client and dispatch them to the corresponding module or component.
\paragraph{\textit{ASOS Activation}} Offers the functionality for the activation and deactivation of the ASOS service.
\paragraph{\textit{Health Data analyzer}} Offers functionality to extrapolate the critical health parameters for every Individual User;
\paragraph{\textit{Emergency Handler}} Responsible to handle critical health conditions based on the data published by the \textit{Data collector/dispatcher}
%T4R---------------------------------------------------------------------------------------------------------
\subsection{Track4Run}
\label{subsect:T4RComponentView}
\begin{figure}[H]
\caption{Track4Run Component View}
\centering
\includegraphics[width = \textwidth]{sections/architecturalDesign/T4RDiagram.png}
\end{figure}
\paragraph{\textit{T4R router}} Validate the requests received from the client and dispatch them to the corresponding module or component.
\paragraph{\textit{T4R Activation}} Offers the functionality for the activation and deactivation of the T4R service.
\paragraph{\textit{Run Manager}} Provides functionality to create, cancel and enrol in runs.
%Complete-------------------------------------------------------------------------------------------------
\subsection{Full system}
\begin{figure}[H]
\label{fig:ComponentDiagram}
\caption{Complete Component View}
\centering
\includegraphics[width = \textwidth]{sections/architecturalDesign/ComponentDiagram.png}
\end{figure}
\paragraph{\textit{Data Managing}} From a more high level point of view, the backbone provides services to retrive the Individual Users live data. \\
This makes the red components and modules of the architecture the backbone, collecting and dispatching data, while the other subsystems can handle their unique authorization condition: D4H authorizing data dispatching based on approved requests, ASOS on the activation of the service and T4R on the enrollement in competitions. \\
This way all subsystem can work independently from each other.%EntityRelationshipDiagram-------------------------------------------------------------------------------------------------
\subsection{Entity Relationship Diagram}
The following section provides a conceptual representation of the model.
\begin{figure}[H]
\caption{Entity Relationship Diagram}
\centering
\includegraphics[width = \textwidth]{sections/architecturalDesign/entityRelationshipDiagram.png}
\end{figure}
\paragraph{\textit{Tables}} 
\begin{itemize}
\item \textbf{\textit{User}}(\underline{ID}, Name, Email, Password)
\item \textbf{\textit{TPU}}(\underline{ID}, Name, Email, Password)
\item \textbf{\textit{IU}}(\underline{ID}, Name, Email, Password, Age, Gender, Address, Weight, Height)
\item \textbf{\textit{Data}}(\underline{IU}, \underline{Time}, Location, Heartbeat, Blood pressure, Oxygen saturation)
\\
\item \textbf{\textit{Individual Request}}(\underline{Request Identification Number}, IU, TPU, Time, State, Subscription?)
\item \textbf{\textit{Anonym Request}}(\underline{Request Identification Number}, TPU, Time, State, Subscription?, Min Age, Max Age, Min Weight, Max Weight, Min Height, Max Height, Gender, Address)
\\
\item \textbf{\textit{Run}}(\underline{Run Identification number}, TPU, IU, Path, Start Time, End Time, Min Age, Max Age, Min participants, Max Participants)
\item \textbf{\textit{Run Result}}(\underline{Run Identification number}, \underline{IU}, Lenght, Time, Arrival Position)
\end{itemize}
%ModelInteractionDiagram-------------------------------------------------------------------------------------------------
\subsection{Model Interaction Diagram}
The following diagram show a different representation of the model to better highlight its interaction with the application server. For each subsystem module that was connected to the DBMS in \ref{fig:ComponentDiagram} is shown its relationship with the module. 
\begin{figure}[H]
\caption{Model Interaction Diagram}
\label{fig:ModelInteractionDiagram}
\centering
\includegraphics[width = \textwidth]{sections/architecturalDesign/modelInteractionDiagram.png}
\end{figure}
\clearpage
%-------------------------------------------------------------------------------------------------------------
\section{Deployment view}
As stated in the previous sections the system is composed by the two clients, one hosted on a web browser and the other on moblie application. They both rely on the application server while the former also interacts with the web server which host the web application.
The application server provide the logic of the system and interacts with the database server which hosts the data layer of the system.
\begin{figure}[H]
\caption{Deployment view}
\centering
\includegraphics[width = \textwidth]{sections/architecturalDesign/DeploymentView.png}
\end{figure}
%-------------------------------------------------------------------------------------------------------------
\section{Runtime view}
\begin{figure}[H]
\caption{IU Registration}
\centering
\includegraphics[width = \textwidth]{sections/architecturalDesign/IUregistration.png}
\end{figure}
\begin{figure}[H]
\caption{Data Requests}
\centering
\includegraphics[width = \textwidth]{sections/architecturalDesign/dataRequests.png}
\end{figure}
\begin{figure}[H]
\caption{Emergency call}
\centering
\includegraphics[width = \textwidth]{sections/architecturalDesign/emergencyCall.png}
\end{figure}
%-------------------------------------------------------------------------------------------------------------
\section{Component interfaces}
The next diagram shows the most important methods of the components interfaces which, for clarity, are named in figure \ref{fig:ComponentDiagram} tracing the components names.\\
The routers gather oll the method required to provide the client with the corresponding subsystem services and expose the relative APIs for the clients (for the D4H router also non-human TPUs) to use. \\
An generic interface \textit{Data Receiver} is extended by all the interfaces that use the \textit{Data Dispatcher} service, to receive the updates.
\begin{figure}[H]
\caption{Component Interfaces}
\centering
\includegraphics[width = \textwidth]{sections/architecturalDesign/componentInterfaces.png}
\end{figure}
%-------------------------------------------------------------------------------------------------------------
\section{Selected architectural styles and patterns}
\paragraph{Client/server multi-tier}
The architecture style chosen is a client/server structure with multiple tiers. The presentation layer is diveded between the two clients (IUs and TPUs clients) which are thick clients since they host a branch of the application logic to handle better and faster the system functionalities; namely, to provide the fastest possible emergency response time, the client directly handles critical conditions contacting the emergency service and the backbone handles the dispatching of the IU live data to other components on the client. \\
The application server hosts the logic layer, exposing API the clients to access the subsystem functionalities and, for the D4H router, to non-human TPUs which might access directly through the APIs; The application server is divided in four subsystems, each handling a piece of logic: a backbone, handling the core logic, storing data and user authorization, and providing interfaces to other subsystem to use its functionality, while the other subsystem independently handle the functionalities of the three services offered: D4H, ASOS and T4R. \\
The database server host the data layer and all the subsystems on the application server independtly interact with it.
\\
This will make for a modular software, enabling a fairly independent implementation and testing of each subsystem; Morover it, alongside the tiered structure, will improve scalability and maintainability.

\paragraph{Event based paradigm}
The backbone, namely the Data Dispatcher components, is an event-based subsystem that handles the dispatch of live data through the system. Live data collected by the Data Collector/Dispatcher serves as the event, broadcasted to all registered components. While introducing potential scalability problems, it simplify the addition of the other subsystem.