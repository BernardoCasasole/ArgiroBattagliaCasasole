\section{Purpose}
This document means to provide developers and stakeholders with detailed information regarding the architecture of the software.   
A particular effort has been devoted to testing plan and integration testing plan, since the Integration Test Plan Document is not required. 
%-------------------------------------------------------------------------------------------------------------
\section{Scope}
The Data4Help System means to provide services to authenticated users only. Those services are addressed to both:
\begin{itemize}
\item	Individual Users
\item Third parties Users
\end{itemize}
To dispatch specific functionalities to the user they are reserved, the Data4Help System avails itself of:
\begin{itemize}
\item a Mobile Application, reserved to individual users
\item a Web Page, reserved to third party users
\end{itemize}
The Mobile Application, using the GPS location provided by the smartphone, allows the individual user to:
\begin{itemize}
\item check his own health parameters (measured by a smartwatch)
\item enable and disable additional services (AutomatedSOS and Track4Run)
\item give or deny authorization to every third party to access health data about himself
\end{itemize}
The Data4Help System handles both data of the past and real time ones.
The Web Page allows the Third-party user to:
\begin{itemize}
\item make requests for statistical data of the past or real time
\item make requests for individual data of the past or real time (the requests are forwarded to the individual user)
\item organize and watch run competitions
\end{itemize}
This factorization allows the system to be accurate in providing every user with all and only resources he has the right to access: authentication and authorization processes rely on the access control. \\ 
\\
The necessity to use a mobile app could prevent third parties from choosing Data4Help over other services of data collection: the Data4Help Web Page can be easily accessed from a browser hosted on a computer or a mobile. 

%-------------------------------------------------------------------------------------------------------------
\section{Definitions}
\begin{itemize}
\item \textit{Data4Help System}: the whole system, offering Data4Help, AutomateSOS and Track4Run services.
\item \textit{User}: a person or an entity, that has registered;
\item \textit{Individual User}: every registered person from whom the system collects data; 
\item \textit{Third-Party User}:every entity registered with the purpose to request data for external use;
\item \textit{non-human Third-Party User}: a software that access to the offered D4H services through the exposed APIs
\item \textit{Live Data}: the data about a IU produced in real time.
\item \textit{Stored Data}: the data about a IU collected so far.
\item \textit{Data Request}: a request for data made from a TPU.
\item \textit{Stored Data Request}: a data request for stored data.
\item \textit{Subscription Request}: a request for subscribing to newly generated data.
\end{itemize}
\section{Acronyms}
\begin{itemize}
\item API: Application Programming Interface
\item IU: Individual User
\item TPU: Third-party User
\item	D4H: Data4Help
\item	ASOS: AutomatedSOS
\item T4R: Track4Run
\item UX: User experience
\item REST: REpresentational State Transfer
\item EENA: European Emergency Number Association
\item PSAPs: Public Safety Answering Points
\item NG112: Next Generation 112
\item DRT: Database Response Time
\end{itemize}

\section{Abbreviations}
\begin{itemize}
\item Gn: n-goal
\item Dn: n-Domain assumption
\item Rn: n-Requirement
\end{itemize}

%-------------------------------------------------------------------------------------------------------------
\section{Revision history}
\begin{itemize}
\item \textbf{v0.1 - 27/11/18} Document created
\item \textbf{v0.2 - 30/11/18} Component view
\item \textbf{v0.3 - 2/12/18} Model diagrams, User interface and High level overview
\item \textbf{v0.4 - 8/12/18} Architectural patterns, interfaces, deployment, high level architecture review
\item \textbf{v0.5 - 10/12/18} Implementation, integration and testing 
\item \textbf{v1.0 - 10/12/18} Requirement traceability, references, introduction, general revision
\item \textbf{v1.1 - 14/12/18} Nomenclature corrections, grammar corrections
\end{itemize}
%-------------------------------------------------------------------------------------------------------------
\section{Document Structure}
\paragraph{\hyperref[sect:introduction]{Introduction}} \mbox{}\\
This section means to present briefly the software and the world It\textquotesingle s going to live in. The terminology that is going to be used through the document is specified.
\paragraph{\hyperref[sect:architecturalDesign]{Architectural Design}} \mbox{}\\
This section illustrates:
\begin{itemize}
\item high level components and their interaction
\item main components the system is divided into and their interaction
\item diagrams reporting entities relationship
\item different representation of the model that highlights the interaction with server, it’s formalization in the deployment view
\item sequence diagram showing the runtime behavior 
\item component interfaces 
\item patterns.
\end{itemize}
\paragraph{\hyperref[sect:userInterfaceDesign]{User	 Interface Design}}\mbox{}\\
In this section each user interface (presented in RASD) is reported and accurately explained. The interaction among interfaces clarifies the path to reach every interface starting from any other. 
\paragraph{\hyperref[sect:requirementsTraceability]{Requirements Traceability}}\mbox{}\\
This section means to map all the requirements with the corresponding design components. 
\paragraph{\hyperref[sect:implementationIntegrationTestplan]{Implementation, Integration and Test plan}}\mbox{}\\
This section reports how the implementation of components has been designated, including: the implementation order, the integration order, components testing, components integration testing. Details about implementation and testing are provided.  
\paragraph{\hyperref[sect:effort]{Effort Spent}}\mbox{}\\
This section reports, with a tabular representation, the effort spent by each member of the group. 
\paragraph{\hyperref[sect:references]{References}}\mbox{}\\
This section is a list of documents and web sites consulted in order to realize the Design Document. 


